\section*{Curry-Huhn Suppe}
\addtocontents{toc}{\protect\contentsline{section}{Curry-Huhn Suppe}{}{}}
\vspace{2cm}
\begingroup
	\parfillskip=0pt
\begin{minipage}[t][][t]{0.33\linewidth}
{{\large \textbf{Zutaten}}}\\\\
\begin{tabular}{ll}
	\multicolumn{2}{l}{\textbf{2 Portionen}}\\
1&Zwiebel\\
1&Zehe Knoblauch\\
&Ingwer (daumengroß)\\
2\,TL&Curry-Paste\\
0,5\,TL&Kurkuma\\
$\sim$ 0,5\,L&Gemüsebrühe\\
1&Hünchenbrust\\
1& Limette\\
2&Karotten\\
&Reisnudeln\\
Dose&Kokosmilch\\
&Sojasauce\\
kl. Dose&Mais\\
1&Frühlingszwiebel\\
\end{tabular}
\end{minipage}
\hfill
\begin{minipage}[t][][t]{0.66\linewidth}
Zwiebel würfeln und anschwitzen. Derweil Ingwer und Knoblauch fein hacken und anschließend zur Zwiebel geben. Curry-Paste und Kurkuma zugeben und andünsten, anschließend mit Gemüsebrühe und Kokosmilch ablöschen. Saft der Limette hinzugeben. Die Hünchenbrust darin für 15\,min pochieren. Danach herausnehmen und klein reißen. Karotten schälen, in dünne Stifte schneiden und in die Brühe geben. Die Reisnudeln können nun entweder mit in die Brühe gegeben werden oder seperat gekocht und nach Bedarf in die Teller gegeben werden. Mit Sojasauce abschmecken. Frühlingszwiebel scheiden und zusammen mit dem Hühnchen und dem Mais in die Brühe geben.
\end{minipage}
\clearpage