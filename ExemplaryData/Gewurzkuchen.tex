\section*{Gewürzkuchen}
\addtocontents{toc}{\protect\contentsline{section}{Gewürzkuchen}{}{}}
\vspace{2cm}
\begingroup
	\parfillskip=0pt
\begin{minipage}[t][][t]{0.33\linewidth}
{{\large \textbf{Zutaten}}}\\\\
\begin{tabular}{rl}
125\,g&Margarine\\
295\,g&Zucker\\
320\,g&Mehl\\
125\,g&geriebene\\
&Zartbitterschokolade\\
4&Eier\\
&Vanille\\
2\,TL&gemahlene Nelke\\
2\,TL&gemahlenen Zimt\\
\textonehalf&Muskatnuss\\
\textonequarter\,L&Milch\\
1\,Pkt.&Backpulver\\
150\,g&Puderzucker\\
1&Zitrone\\
\end{tabular}
\end{minipage}
\hfill
\begin{minipage}[t][][t]{0.66\linewidth}
Fett und Zucker schaumig rühren, dann die Eier zufügen. (Am besten alles vorher aus dem Kühlschrank nehmen, damit es nicht zu kalt ist.) Mehl, Backpulver und Gewürze trocken mischen, mit Schokolade und Milch unter die Fett/Zucker/Eier Masse rühren  (ist ziemlich flüssig).\\
Teig in eine Form füllen. Bei 180\,°C (Umluft 160\,°C) 60-80\,min backen.\\\\
\textbf{Zitronenguss:}\\
"Dafür hab ich leider keine exakten Angaben, das probieren wir selber immer aus. Man benötigt etwa 6\,EL (weniger) Zitronensaft (ca. 2\,Zitronen) und etwa 100 bis 150\,g (vielleicht sogar mehr) Puderzucker. Am besten erstmal einen Teil des Saftes in den Zucker rühren, nicht umgekehrt. Er sollte eine dickflüssige streichfähige Konsistenz haben. Je dünner, desto öfter muss man den Guss aufstreichen. Aber ohne den Guss bringt's der Kuchen nicht, finde ich.\"\ - Karoline Diehl\\

\textbf{Muffins (so hatte es gut geklappt):}\\
Statt dem Päckchen Backpulver nur 3\,TL und noch 0,5\,TL Natron. Zusätzlich ein kleiner Becher Joghurt und etwas mehr Schokolade (20\,g). Weißmehl habe ich mit Vollkornmehl ersetzt und weißen Zucker mit braunem Zucker. Das ergaben 42\,Muffins. - Angelika Diehl\end{minipage}
\vfill\noindent
Rezept von Klara Korrell\\
\clearpage